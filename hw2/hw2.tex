\documentclass{article}
\usepackage{fancyhdr}
\usepackage{lipsum}  
\usepackage{listings} 
\usepackage{xcolor}   
\usepackage{amsmath}
\usepackage{enumitem}
\usepackage{graphicx}
\usepackage{caption}
\usepackage{verbatim}

% Define macros for title and author
\newcommand{\thetitle}{STAT 641 \\ Homework 2}
\newcommand{\theauthor}{Keegan Smith}

\title{\thetitle}
\author{\theauthor}

\pagestyle{fancy}
\fancyhf{}  % Clear all header and footer fields
\fancyhead[L]{\nouppercase{\rightmark}}
\fancyhead[C]{\thetitle}  % Title in the center
\fancyhead[R]{\theauthor}  % Your name on the right

\lstset{ %
  backgroundcolor=\color{lightgray},   % choose the background color
  basicstyle=\ttfamily\small,          % size of fonts used for the code
  keywordstyle=\color{blue},           % color for keywords
  commentstyle=\color{green},          % color for comments
  stringstyle=\color{red},             % color for strings
  numbers=left,                        % where to put the line-numbers
  numberstyle=\tiny\color{gray},       % style for line-numbers
  stepnumber=1,                        % the step between two line-numbers
  numbersep=5pt,                       % how far the line-numbers are from the code
  frame=single,                        % adds a frame around the code
  rulecolor=\color{black},             % frame color
  breaklines=true,                     % automatic line breaking
  breakatwhitespace=false,             % automatic breaks should only happen at whitespace
  showspaces=false,                    % don't show spaces in the code
  showstringspaces=false,              % don't show spaces in strings
  showtabs=false,                      % don't show tabs in the code
}

\begin{document}

\maketitle

\section*{Question Group 1}
\section*{1.1}
\begin{enumerate}
\item The cumulative distribution function $F(z)$ can be given by: \\
\[
F(z) = \int_{-\infty}^zf(x)dx
\]
Thus we have: \\
\begin{align*}
F(z) &= \int_{-\infty}^0f(x)dx + \int_{0}^{z}f(x)dx \\
F(z) &= 0 + \int_{0}^{z}\lambda \cdot e^{-\lambda \cdot x}dx \\
F(z) &= (-e^{-\lambda \cdot x})_0^z \\
F(z) &= -e^{-\lambda \cdot z} + 1
\end{align*}
\item Given $F(z)$ from the above: \\
\begin{align*}
F(z) &= 0.5 \\
-e^{-\lambda \cdot z} + 1 &= .5 \\
-e^{-\lambda \cdot z} &= -.5 \\
-\lambda \cdot z &= \ln(.5) \\
z &= -\frac{\ln(.5)}{\lambda}
\end{align*}
\end{enumerate}
\section*{1.2}
\begin{enumerate}
\item The z stat for 60 is: \\
\begin{align*}
z &= \frac{60 - 50}{10} \\
z &= 1 \\
\end{align*}
Using a calculator, the cdf of z is .8413
\item we are trying to find the value such that cdf($z$) = .95 where $z$ is the z statistic: \\
\begin{align*}
cdf(z) &= .95 \\
z &= icdf(.95) \\
\frac{x - \mu}{\sigma} &= icdf(.95) \\
x &= icdf(.95) \cdot \sigma + \mu \\
x &\approx 66.4485
\end{align*} 
\end{enumerate}
\section*{Question Group 2}
\section*{2.1}
\begin{enumerate}
\item $\mu$ is a location variable if: \\
\[
f_W(w) = f_Y(w + \mu)
\]
does not depend on $\mu$ where $W = Y - \mu$. \\
For $\xi \neq 0$ we have: \\
\begin{align*}
f_W(w) &= f_Y(w + \mu) \\
&= \frac{1}{\sigma} \cdot (1 + \xi \cdot \frac{(z + \mu) - \mu}{\sigma})^{-1 - \frac{1}{\xi}}\mbox{exp}(-(1 + \xi \cdot \frac{(z + \mu) - \mu}{\sigma})^{-\frac{1}{\xi}}) \\
&= \frac{1}{\sigma} \cdot (1 + \xi \cdot \frac{z}{\sigma})^{-1 - \frac{1}{\xi}}\mbox{exp}(-(1 + \xi \cdot \frac{z}{\sigma})^{-\frac{1}{\xi}}) \\
\end{align*}
The pdf of $f_W(w)$ does not depend on $\mu$ so $\mu$ is a location variable when $\xi \neq 0$ \\
For $\xi = 0$ we have: \\
\begin{align*}
f_W(w) &= f_Y(w + \mu) \\
&= \frac{1}{\sigma} \cdot \mbox{exp}(-\frac{(z + \mu) - \mu}{\sigma})\mbox{exp}(\mbox{-exp}(-\frac{(z + \mu) - \mu}{\sigma})) \\
&= \frac{1}{\sigma} \cdot \mbox{exp}(-\frac{z}{\sigma})\mbox{exp}(\mbox{-exp}(-\frac{z}{\sigma}))
\end{align*}
again the pdf of $f_W(w)$ does not depend on $\mu$ so $\mu$ is a location variable when $\xi = 0$. Therefore $\mu$ is a location variable. \\
$\sigma$ is a scaling variable if: \\
\[
f_W(w) = \sigma \cdot f_Y(\sigma w)
\]
does not depend on $\sigma$ where $W = \frac{Y}{\sigma}$. \\
Since $\mu$ is a location variable, let $Y = Z - \mu$ \\
For $\xi \neq 0$ we have: \\
\begin{align*}
f_W(w) &= \sigma \cdot f_Y(\sigma w) \\
&= \sigma \cdot \frac{1}{\sigma} \cdot (1 + \xi \cdot \frac{y \cdot \sigma}{\sigma})^{-1 - \frac{1}{\xi}}\mbox{exp}(-(1 + \xi \cdot \frac{y \cdot \sigma}{\sigma})^{-\frac{1}{\xi}}) \\
&= (1 + \xi \cdot y)^{-1 - \frac{1}{\xi}}\mbox{exp}(-(1 + \xi \cdot y)^{-\frac{1}{\xi}}) \\
\end{align*}
The final function does not depend on $\sigma$ so $\sigma$ is a scaling variable when $\xi \neq 0$ \\
For $\xi = 0$ we have: \\
\begin{align*}
f_W(w) &= \sigma \cdot f_Y(\sigma w) \\
&= \sigma \cdot \frac{1}{\sigma} \cdot \mbox{exp}(-\frac{y \cdot \sigma}{\sigma})\mbox{exp}(\mbox{-exp}(-\frac{y \cdot \sigma}{\sigma})) \\
&= \mbox{exp}(-y)\mbox{exp}(\mbox{-exp}(-y))
\end{align*}
The final function does not depend on $\sigma$ so $\sigma$ is a scaling variable for $\xi = 0$. $\sigma$ is a scaling variable both when $\xi = 0$ and $\xi \neq 0$ so $\sigma$ is a scaling variable for the family. \\
\item The quantile function is simply the inverse of the cdf: \\
\[
Q(p) = F^{-1}(p)
\]
Thus we have: \\
\begin{align*}
p &= \mbox{exp}(-(1 + \xi(\frac{z - \mu}{\sigma}))^{-\frac{1}{\xi}}) \\
\ln(p) &= -(1 + \xi(\frac{z - \mu}{\sigma}))^{-\frac{1}{\xi}} \\
(-\ln(p))^{-\xi} &= 1 + \xi(\frac{z - \mu}{\sigma}) \\
(-\ln(p))^{-\xi} - 1 &= \xi(\frac{z - \mu}{\sigma}) \\
\frac{(-\ln(p))^{-\xi} - 1}{\xi} \cdot \sigma &= z - \mu \\
z &= \frac{(-\ln(p))^{-\xi} - 1}{\xi} \cdot \sigma + \mu
\end{align*}
Thus the quantile function for $\xi \neq 0$ is: \\
\[
Q(p) = \frac{(-\ln(p))^{-\xi} - 1}{\xi} \cdot \sigma + \mu
\]
for when $\xi = 0$: \\
\begin{align*}
p &= \mbox{exp}(-\mbox{exp}(-\frac{z - \mu}{\sigma})) \\
\ln(p) &= -\mbox{exp}(-\frac{z - \mu}{\sigma}) \\
\ln(-\ln(p)) &= -\frac{z - \mu}{\sigma} \\
-\sigma \cdot \ln(-\ln(p)) + \mu &= z \\
\end{align*}
Thus the quantile function for $\xi = 0$ is: \\
\[
Q(p) = -\sigma \cdot \ln(-\ln(p)) + \mu
\]
\item The probability that a value is greater than $x$ is given by: \\
\[
P(X > x) = 1 - F(x)
\]
Thus we have: \\
\begin{align*}
P(X > 12) &= 1 - F(12) \\
&= 1 - \mbox{exp}(-(1 + \frac{1}{2} \cdot \frac{12 - 10}{2})^{-\frac{1}{\frac{1}{2}}}) \\
&\approx .3588 \\
\end{align*}
\item We simply need to evaluate $Q(.25)$: \\
\begin{align*}
Q(.25) &= \frac{(-\ln(.25))^{-\frac{1}{2}} - 1}{\frac{1}{2}} \cdot 2 + 10 \\
&\approx 9.397
\end{align*}
\end{enumerate}
\section*{2.2}
\begin{enumerate}
\item $\beta$ acts as a scaling parameter if: \\
\[
f_W(w) = \beta \cdot f_Y(\beta w)
\]
does not depend on $\beta$ where $W = \frac{Y}{\beta}$. \\
Thus we have: \\
\begin{align*}
f_W(w) &= \beta \cdot \frac{1}{\beta^{\alpha}\Gamma(\alpha)}(y \cdot \beta)^{\alpha - 1} \cdot e^{\frac{-\beta \cdot y}{\beta}} \\
f_W(w) &= \beta \cdot \frac{1}{\beta \cdot \Gamma(\alpha)}y^{\alpha - 1}\cdot e^{-y} \\
f_W(w) &= \frac{1}{\Gamma(\alpha)}y^{\alpha - 1}\cdot e^{-y} \\
\end{align*}
The final function doesn't rely on $\beta$ so $\beta$ is a scaling variable for $y \geq 0$. The case for $y < 0$ is trivial since the pdf is just 0. So $\beta$ is a scaling variable for this family. \\
\item We now have: \\
\[
f(y) = \frac{1}{\frac{1}{\theta}^{\alpha}\Gamma(\alpha)}(y)^{\alpha - 1} \cdot e^{\frac{-y}{\frac{1}{\theta}}} 
\]
Instead of using $W = \frac{Y}{\theta}$, we will use $W = \frac{Y}{\theta^{-1}}$: \\
\begin{align*}
f_W(w) &= \frac{1}{\theta} \cdot \frac{1}{\frac{1}{\theta}^{\alpha}\Gamma(\alpha)}(y \cdot \theta^{-1})^{\alpha - 1} \cdot e^{\frac{-(y \cdot \theta^{-1})}{\frac{1}{\theta}}} \\
f_W(w) &= \frac{1}{\theta} \cdot \frac{1}{(\theta^{-1})\Gamma(\alpha)}(y^{\alpha - 1}) \cdot e^{-y} \\
f_W(w) &= \frac{1}{\Gamma(\alpha)}(y^{\alpha - 1}) \cdot e^{-y} \\
\end{align*}
The final function doesn't depend on $\theta$ so $\theta$ is a sclaing variable (again $y < 0$ is trivial). 
\end{enumerate}
\section*{2.3}
\begin{enumerate}
\item Assuming the call rate follows a Poisson distribution with $\lambda = 12$ then we have: \\
\[
P(Y = y) = \frac{e^{-12} \cdot 12^y}{y!}
\]
\begin{align*}
P(Y \geq 3) &= 1 - P(Y \leq 2) \\
&= 1 - P(Y = 0) - P(Y = 1) - P(Y = 2) \\
&= 1 - e^{-12} - 12e^{-12} - 72e^{-12} \\
&= 1 - 85 \cdot e^{-12} \\
&= 0.9995 \\
\end{align*}
\item Since the rate of calls is constant, the distribution for the calls in a week also follows a poisson distribution with $\lambda = 12 \cdot 7 = 84$ \\
Thus we have: \\
\[
P(Y = y) = \frac{e^{-84} \cdot 84^y}{y!}
\]
\begin{align*}
P(Y \geq 3) &= 1 - P(Y \leq 2) \\
&= 1 - P(Y = 0) - P(Y = 1) - P(Y = 2) \\
&= 1 - e^{-84} - 84e^{-84} - 3528e^{-84} \\
&\approx 1.0000 \\
\end{align*}
\end{enumerate}
\section*{2.4}
\begin{enumerate}
\item The cdf for an exponential distribution with mean wait time of 10 minutes is given by: \\
\[
F(t) = 1 - e^{-\frac{1}{10} \cdot t}
\]
So we have: \\
\begin{align*}
F(5) &= 1 - e^{-\frac{1}{10} \cdot 5} \\
&= 1 - e^{-\frac{1}{2}} \\
&\approx 0.3935
\end{align*}
\item The probability that time is greater than 15 is $1 - F(15)$
\begin{align*}
1 - F(15) &= 1 - (1 - e^{-\frac{1}{10} \cdot 15}) \\
&= 1 - (1 - e^{-\frac{3}{2}}) \\
&= e^{-\frac{3}{2}} \\
&\approx 0.2231
\end{align*}
\item The probability that a call will occur in the next 5 minutes is 0.3935 since the amount of time between events does not depend on the amount of time that has passed. \\
\end{enumerate}
\section*{Problem 3}
\section*{3.1}
\begin{enumerate}
\item This is an exponential ($\beta$) distribution \\
\item This is a gamma distribution with parameters ($\alpha_2 + \alpha_3 + \alpha_4$, $\beta$)\\
\item This is a gamma distribution with parameters ($\alpha_5$, $5 \cdot \beta_5$) \\
\item This is a chi-squared distribution with parameter $v = 2 \cdot \alpha_6$ \\
\item This is a maxwell distribution \\
\end{enumerate}
\section*{3.2}
\begin{enumerate}
\item the exponential cdf is: \\
\[
1 - e^{-\lambda \cdot y}
\]
In this case $\lambda = 2$. \\
Thus we have: \\
\begin{align*}
.48 &= 1 - e^{-\lambda \cdot y} \\
.52 &= e^{-\lambda \cdot y} \\
\ln(.52) &= -\lambda \cdot y \\
y &= \frac{-\ln(.52)}{2} \\
&= 0.3269
\end{align*}
\item The cdf for geometric distribution is:\\
\[
1 - (1 - p)^{i}
\]
where $p=.5$ in our case:\\
\begin{align*}
.48 &= 1 - (1 - p)^{i} \\
.52 &= (1 - p)^i \\
\log_{.5}.52 &= i\\
i &= 0.9434
\end{align*}
\item I refuse to do this by hand. Here is the python code using scipy: \\
\begin{verbatim}
import numpy as np
from scipy.stats import binom
value = .48
n = 15
p = .6
i = 0
total_prob = 0
while(True):
    total_prob += binom.pmf(i, n, p)
    if total_prob >= value:
        print("result: ", i)
        break;
    i += 1
\end{verbatim}
The result is 9 \\
\item Again: \\
\begin{verbatim}
total_prob = 0
i = 0
while(True):
  total_prob += poisson.pmf(i, 4)
  if(total_prob >= value):
      print("result: ", i)
      break;
  i += 1
\end{verbatim}
The result is 4. \\
\item For uniform (5, 10) This should simply be $5 + 5 \cdot .48 = 7.4$
\end{enumerate}
\section*{question group 4}
\begin{enumerate}
\item binomial \\
\item exponential \\
\item poisson \\
\item geometric \\
\item binomial \\
\item gamma \\
\item exponential \\
\item lognormal \\
\item poisson \\
\item lognormal \\
\item normal \\
\item hypergeometric \\
\item weibull \\
\item lognormal \\
\item gamma \\
\item poisson
\end{enumerate}
\end{document}
\documentclass{article}
\usepackage{fancyhdr}
\usepackage{lipsum}  
\usepackage{listings} 
\usepackage{xcolor}   
\usepackage{amsmath}
\usepackage{enumitem}
\usepackage{graphicx}
\usepackage{caption}
\usepackage{verbatim}
\usepackage{physics}
% Define macros for title and author
\newcommand{\thetitle}{STAT 641 \\ Homework 7}
\newcommand{\theauthor}{Keegan Smith}

\title{\thetitle}
\author{\theauthor}

\pagestyle{fancy}
\fancyhf{}  % Clear all header and footer fields
\fancyhead[L]{\nouppercase{\rightmark}}
\fancyhead[C]{\thetitle}  % Title in the center
\fancyhead[R]{\theauthor}  % Your name on the right

\lstset{ %
  backgroundcolor=\color{lightgray},   % choose the background color
  basicstyle=\ttfamily\small,          % size of fonts used for the code
  keywordstyle=\color{blue},           % color for keywords
  commentstyle=\color{green},          % color for comments
  stringstyle=\color{red},             % color for strings
  numbers=left,                        % where to put the line-numbers
  numberstyle=\tiny\color{gray},       % style for line-numbers
  stepnumber=1,                        % the step between two line-numbers
  numbersep=5pt,                       % how far the line-numbers are from the code
  frame=single,                        % adds a frame around the code
  rulecolor=\color{black},             % frame color
  breaklines=true,                     % automatic line breaking
  breakatwhitespace=false,             % automatic breaks should only happen at whitespace
  showspaces=false,                    % don't show spaces in the code
  showstringspaces=false,              % don't show spaces in strings
  showtabs=false,                      % don't show tabs in the code
}

\begin{document}

\maketitle
\section*{Problem 1}
\begin{enumerate}
\item independent samples \\
\item matched pairs \\
\item matched pairs \\
\item independent samples \\
\item matched pairs \\
\item independent samples
\end{enumerate}
\section*{Problem 2}
First we will apply the two sample t-test to this problem: \\
we have $n = m = 15$. The average before is 120.8667 and the average after is 110.8. let $\bar{X} = \mbox{sample mean before}$ and $\bar{Y} = \mbox{sample mean after}$. Let $\hat{\theta} = \bar{Y} - \bar{X}$ and $\theta = \mu_Y - \mu_X$. \\
Our null hypothesis is $\theta \geq 0$ and our alternative hypothesis is $\theta < 0$. \\
The pooled standard deviation is: \\
\begin{align*}
s_p &= \sqrt{\frac{(n-1)s_1^2 + (m - 1)s_2^2}{n + m - 2}} \\
&= \sqrt{\frac{(14)4.240395^2 + (14)5.427443^2}{28}} \\
&= 4.8702200809755
\end{align*}
We then have our test statistic: \\
\begin{align*}
T &= \frac{\hat{theta} - \theta_0}{\sqrt{\frac{s^2}{n} + \frac{s^2}{m}}} \\
&= \frac{-10.067 - 0}{\sqrt{\frac{4.87^2}{15} + \frac{4.87^2}{15}}} \\
&= -5.6611
\end{align*}
our degrees of freedom is $15 + 15 - 2 = 28$, so from the table we get that the: \\
\[
P(T < T_obs) = tcdf(-5.6611, 28) \approx 2.29 \cdot 10^{-6}
\]
this p value is much less than $\alpha = .05$, so we reject the null hypothesis. \\
Using the wilcoxon rank sum test, we will let $W_1 = \mbox{sum of before ranks}$ and $W_2 = \mbox{sum of after ranks}$. \\
We get $W_1 = 516$ and $W_2 = 216$. So our p value is: \\
\[
p = G(W_1) = pwilcox(w_1 - n(n+1) / 2, n, m) = 0.004937411
\]
Again the p-value is much less than $\alpha = .05$ so we reject the null hypothesis. \\
For a paired t-test to be used, we must have: 
\begin{enumerate}
\item $D$ is normally distributed where $D_i = X_i - Y_i$: using the shapiro wilkes test we get a p value of 0.5105, suggesting that indeed $D$ is normally distributed. \\
\item $D_i$s are independent: this depends on how exactly the study was carried out, but it would seem that one patient's blood sugar results would not impact another patient's blood sugar results, so I think it is safe to assume that this condition holds. 
\end{enumerate}
given the above, it would seem that the paired t-test is suitable for this problem. \\
\section*{Problem 3}
\begin{enumerate}
\item We have the null hypothesis: \\
\[
H_o: p_1 - p_2 = 0
\]
where $p_1$ is the proportion of population 1 which has the trait, and $p_2$ is the proportion of population 2 which has the trait. We have the alternative hypothesis: \\
\[
H_A: p_1 - p_2 \neq 0
\]
\item We have the test statistic: \\
\begin{align*}
Z &= \frac{\hat{p_1} - \hat{p_2}}{\sqrt{\hat{p}(1 - \hat{p})(\frac{1}{n_1} + \frac{1}{n_2})}} \\
\end{align*}
where: \\
\[
\hat{p} = \frac{n_1 \cdot \hat{p_1} + n_2 \cdot \hat{p_2}}{n_1 + n_2} = \frac{100 \cdot .4 + 100 \cdot .3}{200} = .35
\]
Plugging everything into our test statistic formula: \\
\begin{align*}
Z &= \frac{.4 - .3}{\sqrt{.35(1 - .35)(\frac{1}{100} + \frac{1}{100})}} \\
&\approx 1.4825
\end{align*}
Thus we have the p value: \\
\[
P(\abs{Z} \geq \abs{Z_o}) = 2(1 - \Phi(\abs{Z_o})) = 2(1 - \Phi(1.4825)) = 2(1 - .9306) = .1388
\]
$.1388 > \alpha = .05$ so we fail to reject the null hypothesis. There is not significant evidence to suggest that the proportion of people with the trait differs between the two populations. \\
\end{enumerate}
\section*{Problem 4}
\begin{enumerate}
\item we want to test whether or not the training program had an overall effect on productivity. To check this we need to ask whether or not the proportion of productive people who were hindered by this is different than the proportion of unproductive people that were helped by this. So in other words, we have the null hypothesis: \\
\[
H_0: p_{1,2} = p_k{2,1}
\]
and the alternative hypothesis: \\
\[
H_A: p_{1,2} \neq p_k{2,1}
\]
\item This data is paired (the same participant is giving a response before and after the training program) so we can either use an exact binomial test, or McNemar's test. We will use exact binomial since the sample size is quite small $m = 13 < 30$. $m > 10$ so we can still use a normal approximation for the exact binomial test: \\
\begin{align*}
z &= \frac{n_{1,2} - n_{2,1}}{\sqrt{n_{1,2} + n_{2,1}}} \\
&= \frac{9 - 4}{\sqrt{9 + 4}} \\
&= 1.3868
\end{align*}
so we get a p value of: \\
\begin{align*}
p &= 2 \cdot (1 - \Phi(z)) \\
&= 2 \cdot (1 - .9177) \\
&= 0.1646
\end{align*}
the p-value is greater than .05, so we fail to reject the null hypothesis that the program impacts employee performance.
\end{enumerate}
\section*{Problem 5}
For this problem it seems that we can't use any of the methods outlined in handout 13. We are not testing whether or not each of these proportions are equivalent to one another, but rather testing whether or not the proportion of flowers follows a particular distribution, thus it would be best if we used a goodness of fit test. \\
We can simply use chisquare GOF as outlined in handout 9 where: \\
\[
E = [\frac{9}{16} \cdot 700, \frac{3}{16} \cdot 700, \frac{3}{16} \cdot 700, \frac{1}{16} \cdot 700] = [393.75, 131.25, 131.25, 43.75]
\]
\[
O = [414, 126, 120, 40]
\]
so we have: \\
\begin{align*}
Q &= \frac{(393.75 - 414)^2}{393.75} + \frac{(131.25- 126)^2}{131.25} + \frac{(131.25- 120)^2}{131.25} + \frac{(43.75- 40)^2}{43.75} \\
&= 1.041 + .21 + 0.9643 + 0.3214 \\
&= 2.5367
\end{align*}
So we have the p value: \\
\[
p = 1 - pchisq(2.5367, 3) = 0.4687
\]
This value is much greater than $\alpha = .01$, so we fail to reject the null hypothesis that the flowers follow the distribution 9:3:3:1. \\
\section*{Problem 6}
We are testing for independence and the sample size is quite large (none of the observed values is less than 5) so we can use a pearson chi squared test for independence: \\
First we will compute all of the estimated expected values: \\
\[
E_{1, 1} = \frac{120 \cdot 170}{300} = 68
\]
\[
E_{1, 2} = \frac{120 \cdot 130}{300} = 52
\]
\[
E_{2, 1} = \frac{120 \cdot 170}{300} = 68
\]
\[
E_{2, 2} = \frac{120 \cdot 130}{300} = 52
\]
\[
E_{3, 1} = \frac{60 \cdot 170}{300} = 34
\]
\[
E_{3,2} = \frac{60 \cdot 130}{300} = 26
\]
We then have the test statistic: \\
\begin{align*}
T &= \frac{(68 - 50)^2}{68} + \frac{(52 - 70)^2}{52} + \frac{(68 - 80)^2}{68} + \frac{(52 - 40)^2}{52} + \frac{(34 - 40)^2}{34} + \frac{(26 - 20)^2}{26} \\
&= 4.7647 + 6.2307 + 2.1176 + 2.7692 + 1.0588 + 1.3846 \\
&= 18.3256
\end{align*}
the test statistic is a chi square distribution with $(r - 1)(c - 1) = 2 \cdot 1 = 2$ degrees of freedom, so we have the p value: \\
\[
p = 1 - pchisq(18.3256, 2) = 0.0001048689
\]
The p value is much less than .05, so we reject the null hypothesis that education level and preference type are independent. \\
\section*{Problem 7}
\begin{enumerate}
\item A \\
\item B \\
\item B \\
\item B 
\end{enumerate}
\end{document}
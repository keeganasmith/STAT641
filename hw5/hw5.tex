\documentclass{article}
\usepackage{fancyhdr}
\usepackage{lipsum}  
\usepackage{listings} 
\usepackage{xcolor}   
\usepackage{amsmath}
\usepackage{enumitem}
\usepackage{graphicx}
\usepackage{caption}
\usepackage{verbatim}

% Define macros for title and author
\newcommand{\thetitle}{STAT 641 \\ Homework 4}
\newcommand{\theauthor}{Keegan Smith}

\title{\thetitle}
\author{\theauthor}

\pagestyle{fancy}
\fancyhf{}  % Clear all header and footer fields
\fancyhead[L]{\nouppercase{\rightmark}}
\fancyhead[C]{\thetitle}  % Title in the center
\fancyhead[R]{\theauthor}  % Your name on the right

\lstset{ %
  backgroundcolor=\color{lightgray},   % choose the background color
  basicstyle=\ttfamily\small,          % size of fonts used for the code
  keywordstyle=\color{blue},           % color for keywords
  commentstyle=\color{green},          % color for comments
  stringstyle=\color{red},             % color for strings
  numbers=left,                        % where to put the line-numbers
  numberstyle=\tiny\color{gray},       % style for line-numbers
  stepnumber=1,                        % the step between two line-numbers
  numbersep=5pt,                       % how far the line-numbers are from the code
  frame=single,                        % adds a frame around the code
  rulecolor=\color{black},             % frame color
  breaklines=true,                     % automatic line breaking
  breakatwhitespace=false,             % automatic breaks should only happen at whitespace
  showspaces=false,                    % don't show spaces in the code
  showstringspaces=false,              % don't show spaces in strings
  showtabs=false,                      % don't show tabs in the code
}

\begin{document}

\maketitle
\section*{Problem 1}
\begin{enumerate}
\item the trimmed mean is 886.7853403141361 and the untrimmed mean is 955.3713080168776. This would suggest that the data has some extreme values to the right, and is right skewed. \\
\item the survival function is: \\
\[
S(t) = 1 - F(t)
\]
We can derive $F(t)$ from the pdf, (and since $t$ is a time any probability for $t < 0$ is 0, so we are only considering $t \geq 0$): \\
\begin{align*}
F(t) &= \int_{0}^{t}\lambda e^{-\lambda x} \\
&= (-e^{-\lambda x})_{0}^{t} \\
&= (-e^{-\lambda t} - (-1)) \\
&= 1 - e^{-\lambda t}
\end{align*}
So we have: \\
\begin{align*}
S(t) &= e^{-\lambda t} \\
\ln(S(t)) &= -\lambda t \\ 
\end{align*}
So if $S(t)$ is a good estimate, then the plot $t$ vs $\ln(S(t))$ should be a linear plot with slope: \\
\[
\frac{t}{-\lambda t} = \frac{1}{-\lambda}
\]

\item We have the likelihood function: \\
\begin{align*}
L(\lambda;y) &= \Pi_{k = 0}^{n-1}\lambda e^{-\lambda y_k} \\
&= \lambda^ne^{\sum_{k=0}^{n-1}-\lambda y_k} \\
&= \lambda^ne^{-\lambda \cdot \sum_{k=0}^{n-1} y_k} \\
\end{align*}
The log likelihood function is then: \\
\begin{align*}
\ln(L(\lambda;y)) &= \ln(\lambda^ne^{-\lambda \cdot \sum_{k=0}^{n-1} y_k}) \\
&= \ln(\lambda^n) + (-\lambda \cdot \sum_{k=0}^{n-1} y_k) \\
&= n \cdot \ln(\lambda) -\lambda \cdot \sum_{k=0}^{n-1} y_k
\end{align*}
The derivative of the log likelihood function w.r.t $\lambda$ is: \\
\begin{align*}
\frac{d}{d\lambda} (n \cdot \ln(\lambda) -\lambda \cdot \sum_{k=0}^{n-1} y_k) &= n \cdot \frac{1}{\lambda} - \sum_{k=0}^{n-1} y_k \\
&= \frac{n}{\lambda} - \sum_{k=0}^{n-1} y_k
\end{align*}
Solving for the maximum: \\
\begin{align*}
\frac{n}{\lambda} - \sum_{k=0}^{n-1} y_k &= 0 \\
\lambda &= \frac{n}{\sum_{k=0}^{n-1} y_k} \\
&= \frac{1}{\mu}
\end{align*}
Thus the MLE estimate for $\lambda$ is $\frac{1}{955.3713} \approx 0.001047$
\end{enumerate}
\end{document}
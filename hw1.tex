\documentclass{article}
\usepackage{fancyhdr}
\usepackage{lipsum}  
\usepackage{listings} 
\usepackage{xcolor}   
\usepackage{amsmath}
\usepackage{enumitem}

% Define macros for title and author
\newcommand{\thetitle}{STAT 641 \\ Homework 1}
\newcommand{\theauthor}{Keegan Smith}

\title{\thetitle}
\author{\theauthor}

\pagestyle{fancy}
\fancyhf{}  % Clear all header and footer fields
\fancyhead[L]{\nouppercase{\rightmark}}
\fancyhead[C]{\thetitle}  % Title in the center
\fancyhead[R]{\theauthor}  % Your name on the right

\lstset{ %
  backgroundcolor=\color{lightgray},   % choose the background color
  basicstyle=\ttfamily\small,          % size of fonts used for the code
  keywordstyle=\color{blue},           % color for keywords
  commentstyle=\color{green},          % color for comments
  stringstyle=\color{red},             % color for strings
  numbers=left,                        % where to put the line-numbers
  numberstyle=\tiny\color{gray},       % style for line-numbers
  stepnumber=1,                        % the step between two line-numbers
  numbersep=5pt,                       % how far the line-numbers are from the code
  frame=single,                        % adds a frame around the code
  rulecolor=\color{black},             % frame color
  breaklines=true,                     % automatic line breaking
  breakatwhitespace=false,             % automatic breaks should only happen at whitespace
  showspaces=false,                    % don't show spaces in the code
  showstringspaces=false,              % don't show spaces in strings
  showtabs=false,                      % don't show tabs in the code
}

\begin{document}

\maketitle

\section*{Problem 1}
\begin{enumerate}
\item An observational study only collects information about subjects without giving treatments while an experimental study collects information about individuals who are given a treatment (and who are not given a treatment usually). \\
An example of an observational study is an election poll. \\
An example of an experimental study is a clinical trial. \\
\item Correlation in observational studies can often be misinterpreted as causation. This can be avoided by explicitly stating to your readers that there is no evidence of causation even if the data have strong correlation. \\
\item Probability sampling is when you assign a probability to each sample and select a random sample based on the assigned probabilities. Non-probability sampling is not random, samples are chosen by the researchers, not a random number generator. \\
An example of probability sampling is having a list of patients and flipping a coin for each patient. If heads, the patient is included, if tails the patient is not. \\
An example of non-probability sampling is selecting the next 50 people who walk in the store as samples. \\

\end{enumerate}
\end{document}
\documentclass{article}
\usepackage{fancyhdr}
\usepackage{lipsum}  
\usepackage{listings} 
\usepackage{xcolor}   
\usepackage{amsmath}
\usepackage{enumitem}
\usepackage{graphicx}
\usepackage{caption}
\usepackage{verbatim}

% Define macros for title and author
\newcommand{\thetitle}{STAT 641 \\ Homework 4}
\newcommand{\theauthor}{Keegan Smith}

\title{\thetitle}
\author{\theauthor}

\pagestyle{fancy}
\fancyhf{}  % Clear all header and footer fields
\fancyhead[L]{\nouppercase{\rightmark}}
\fancyhead[C]{\thetitle}  % Title in the center
\fancyhead[R]{\theauthor}  % Your name on the right

\lstset{ %
  backgroundcolor=\color{lightgray},   % choose the background color
  basicstyle=\ttfamily\small,          % size of fonts used for the code
  keywordstyle=\color{blue},           % color for keywords
  commentstyle=\color{green},          % color for comments
  stringstyle=\color{red},             % color for strings
  numbers=left,                        % where to put the line-numbers
  numberstyle=\tiny\color{gray},       % style for line-numbers
  stepnumber=1,                        % the step between two line-numbers
  numbersep=5pt,                       % how far the line-numbers are from the code
  frame=single,                        % adds a frame around the code
  rulecolor=\color{black},             % frame color
  breaklines=true,                     % automatic line breaking
  breakatwhitespace=false,             % automatic breaks should only happen at whitespace
  showspaces=false,                    % don't show spaces in the code
  showstringspaces=false,              % don't show spaces in strings
  showtabs=false,                      % don't show tabs in the code
}

\begin{document}

\maketitle

\section*{Question Group 1}
\section*{1.1}
\begin{enumerate}
\item The CDF of Weibull: \\
\begin{align*}
F(z) &= \int_0^{z}\frac{k}{\lambda} (\frac{x}{\lambda})^{k-1}e^{-(\frac{x}{\lambda})^k} dx\\
\end{align*}
using u substitution where: 
\begin{align*}
u &= (\frac{x}{\lambda})^k \\
du &= \frac{k}{\lambda} \cdot (\frac{x}{\lambda})^{k-1}dx \\
dx &= \frac{du}{\frac{k}{\lambda} \cdot (\frac{x}{\lambda})^{k-1}} \\
\end{align*}
Thus we have: \\
\begin{align*}
F(z) &= \int_0^{(\frac{z}{\lambda})^k}\frac{k}{\lambda} (\frac{x}{\lambda})^{k-1}e^{-u}  \frac{du}{\frac{k}{\lambda} \cdot (\frac{x}{\lambda})^{k-1}}\\
&= \int_0^{ (\frac{z}{\lambda})^k} e^{-u} du \\
&= (-e^{-u})_0^{ (\frac{z}{\lambda})^k} \\
&= (-e^{-(\frac{z}{\lambda})^k} - (-1)) \\
&= (1 - e^{-(\frac{z}{\lambda})^k}) \\
\end{align*}
\item Quantile for p = .5: \\
\begin{align*}
p &= (1 - e^{-(\frac{z}{\lambda})^k}) \\
1 - p &= e^{-(\frac{z}{\lambda})^k} \\
-\ln(1 - p) &= -(\frac{z}{\lambda})^k \\
(-\ln(1 - p))^{\frac{1}{k}} &= (\frac{z}{\lambda}) \\
z &= (-\ln(1 - p))^{\frac{1}{k}} \cdot \lambda \\
z &= (-\ln(1 - .5))^{\frac{1}{3}} \cdot 2 \\
&\approx 1.77
\end{align*}
\item The survival function is: \\
\begin{align*}
S(t) &= 1 - F(t) \\
&= 1 - (1 - e^{-(\frac{z}{\lambda})^k}) \\
& = e^{-(\frac{z}{\lambda})^k} \\
&=  e^{-(\frac{1}{2})^3} \\
&\approx 0.8825
\end{align*}
\item The hazard function is: \\
\begin{align*}
H(t) &= \frac{f(t)}{S(t)} \\
&= \frac{\frac{k}{\lambda} (\frac{t}{\lambda})^{k-1}e^{-(\frac{t}{\lambda})^k}}{e^{-(\frac{t}{\lambda})^k}} \\
&= \frac{\frac{3}{2} (\frac{1}{2})^{3-1}e^{-(\frac{1}{2})^3}}{e^{-(\frac{1}{2})^3}} \\
&\approx 0.375
\end{align*}
\end{enumerate}
\section*{1.2}
\begin{enumerate}
\item CDF of gompertz: \\
\begin{align*}
F(z) &= \int_0^z\eta b e^{bx}e^{-\eta(e^{bx} - 1)}dx \\
\end{align*}
Using u sub where: \\
\begin{align*}
u &= \eta(e^{bx} - 1) \\
du &= \eta b e^{bx}dx \\
dx &= \frac{du}{\eta b e^{bx}} 
\end{align*}
We then have: \\
\begin{align*}
F(z) &= \int_0^{\eta(e^{bz} - 1)}\eta b e^{bx}e^{-u}\frac{du}{\eta b e^{bx}} \\
&= \int_0^{\eta(e^{bz} - 1)}e^{-u}du \\
&= (-e^{-u})_0^{\eta(e^{bz} - 1)} \\
&= (-e^{-(\eta(e^{bz} - 1))} - (-1)) \\
&= 1 - e^{-(\eta(e^{bz} - 1))} \\
\end{align*}
\item The survival function is: \\
\begin{align*}
S(t) &= 1 - F(t) \\
&= 1 -  (1 - e^{-(\eta(e^{bz} - 1))}) \\
&= e^{-(\eta(e^{bz} - 1))} \\
\end{align*}
\end{enumerate}
\end{document}
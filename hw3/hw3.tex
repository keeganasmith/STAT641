\documentclass{article}
\usepackage{fancyhdr}
\usepackage{lipsum}  
\usepackage{listings} 
\usepackage{xcolor}   
\usepackage{amsmath}
\usepackage{enumitem}
\usepackage{graphicx}
\usepackage{caption}
\usepackage{verbatim}

% Define macros for title and author
\newcommand{\thetitle}{STAT 641 \\ Homework 4}
\newcommand{\theauthor}{Keegan Smith}

\title{\thetitle}
\author{\theauthor}

\pagestyle{fancy}
\fancyhf{}  % Clear all header and footer fields
\fancyhead[L]{\nouppercase{\rightmark}}
\fancyhead[C]{\thetitle}  % Title in the center
\fancyhead[R]{\theauthor}  % Your name on the right

\lstset{ %
  backgroundcolor=\color{lightgray},   % choose the background color
  basicstyle=\ttfamily\small,          % size of fonts used for the code
  keywordstyle=\color{blue},           % color for keywords
  commentstyle=\color{green},          % color for comments
  stringstyle=\color{red},             % color for strings
  numbers=left,                        % where to put the line-numbers
  numberstyle=\tiny\color{gray},       % style for line-numbers
  stepnumber=1,                        % the step between two line-numbers
  numbersep=5pt,                       % how far the line-numbers are from the code
  frame=single,                        % adds a frame around the code
  rulecolor=\color{black},             % frame color
  breaklines=true,                     % automatic line breaking
  breakatwhitespace=false,             % automatic breaks should only happen at whitespace
  showspaces=false,                    % don't show spaces in the code
  showstringspaces=false,              % don't show spaces in strings
  showtabs=false,                      % don't show tabs in the code
}

\begin{document}

\maketitle

\section*{Question Group 1}
\section*{1.1}
\begin{enumerate}
\item The CDF of Weibull: \\
\begin{align*}
F(z) &= \int_0^{z}\frac{k}{\lambda} (\frac{x}{\lambda})^{k-1}e^{-(\frac{x}{\lambda})^k} dx\\
\end{align*}
using u substitution where: 
\begin{align*}
u &= (\frac{x}{\lambda})^k \\
du &= k \cdot (\frac{x}{\lambda})^{k-1}dx \\
dx &= \frac{du}{k \cdot (\frac{x}{\lambda})^{k-1}} \\
\end{align*}
Thus we have: \\
\begin{align*}
F(z) &= \int_0^{(\frac{z}{\lambda})^k}\frac{k}{\lambda} (\frac{x}{\lambda})^{k-1}e^{-u}  \frac{du}{k \cdot (\frac{x}{\lambda})^{k-1}}\\
&= \frac{k}{\lambda} \cdot  \int_0^{ (\frac{z}{\lambda})^k} e^{-u} du \\
&= \frac{k}{\lambda} \cdot (-e^{-u})_0^{ (\frac{z}{\lambda})^k} \\
&= \frac{k}{\lambda} \cdot (-e^{-(\frac{z}{\lambda})^k} - (-1)) \\
&= \frac{k}{\lambda} \cdot (1 - e^{-(\frac{z}{\lambda})^k}) \\
\end{align*}
\end{enumerate}
\end{document}